\newpage
%!почему не работает оглавление?
%\tableofcontents
%\newpage
\section{Введение}
В настоящий момент довольно остро стоит вопрос о сохранении тайны связи при использовании электронной почты, чата, социальных сетей и иных электронных средств коммуникаций. В настоящий момент  закон о сохранении тайны связи не охватывает публичные сервисы.\footnote{ Тайна связи, электронная почта и российские суды (\url{http://www.securitylab.ru/blog/personal/emeliyannikov/37733.php})}
Кроме того, опубликованные Эдвардом Сноуденом данные наглядно демонстрируют, что межправительственные системы слежения (созданные для борьбы с терроризмом) используются для достижения экономических и политических целей, и нарушают права граждан на тайну частной жизни и тайну переписки.\\

Целью проекта является создание веб-приложения, демонстрирующего различные возможности по сбору сведений об отдельном человеке с использованием только открытых источников.\footnote{Правительство США предало интернет. Нам надо вернуть его в свои руки     (\url{http://habrahabr.ru/post/192852/} )\footnote{Эдвард Сноуден (\url{http://ru.wikipedia.org/wiki/Сноуден,\_Эдвард})}} 
Особенно интересным представляется создать веб-сервис для автоматизированного анализа страницы в социальной сети с выделением дополнительных сведений о человеке, на основе сведений о его друзьях. Веб-сервис планируется создать в демонстрационный целях т.к. решение о публикации своих данных осуществляется непосредственно человеком.\\

Люди часто недооценивают значение метаданных и комплексного анализа. 
%нужно нормально построить предложение
Под комплексным анализом далее будет подразумеваться сочетание методов, подходов, инструментов по интеллектуальной добыче данных (Data Mining), %так в скобках и писать?
использованию больших объемов данных (Big Data), 
Люди не задумываются о том, что вступая в определенные электронные сообщества их личные данные может сообщить не только владелец аккаунта, но и другие участники сообщества. Причем, чем ближе они знакомы, тем больше данных они могут передать, иногда даже сами того не подозревая. Так же большой интерес вызывает возможность автоматического извлечение фактов из текста на естественном языке. Не так давно исследователь сотрудники IBM Research во главе с Джалал Махмудом (Jalal Mahmud) опублиоквали научную работу в которой демонстрируют возможность опередить местонахождение человека по его постам в Twitter с точностью до 70\% (определяется обычно город или округ).\footnote{Who will retweet this?: Automatically Identifying and Engaging Strangers on Twitter to Spread Information (\url{digital.cs.usu.edu/~kyumin/pubs/lee14iui.pdf‎})}
 Основная идея алгоритма придуманного сотрудиками IBM заключается в том, что само содержание твитов несет в себе информацию о местонахождение и современные  инструменты  позволяют ее извлечь. Так например в посте может быть ссылка на фото или пост в другой социальной сети в которой отмечена гео-информация, кроме того анализируются симантика текста для извлечение фактов, например из текста "Сергей не забудь самовар, встрчаемся в Туле" Можно извлечь следующие факты: Место - Тула, Объект - Сергей. Всю необходимую информацию исследователи извлекают напрямую из Twitter c помощью Streaming API в основном используя GET statuses/firehose \footnote{подробнее см. Twitter Rest API (\url{https://dev.twitter.com/docs/api/1.1}) и публикацию Jalal Mahmud, Jilin Chen, Michelle Zhou, Jeffrey Nichols Who will...ad Information} \\



Данный сервис задуман с целью проверки оценки уровня защищенности персональной информации, которую пользователь оставляет конфиденциальной становясь участником виртуального сообщества, но которая может быть получена в результате анализа косвенных источников. \\

%что я тут хочу сказать? нужно это дальше развить
Данный сервис не является социально опасным по следующим причинам:
\begin{itemize}
\item пользователь сервиса имеет возможность анализа только той страницы, для которой известны данные авторизации;
\item сервис безопасен для пользователя т. к. авторизация происходит по средствам  API  социальной сети и данные авторизации не передаются на сервер приложения; 
\item мировой опыт показывает, что уже созданы куда более мощные средства для анализа данных. Однако, все они являются достоянием специальных служб. Данный сервис является попыткой защитить конечного пользователя, демонстрируя ему часть той информации, которую о нем могут собрать соответствующие службы.
\end{itemize}

\section{Основной функционал приложения}
Обязательный функционал позволит определить пол, возраст, ВУЗ некоторого человека в социальный сети Вконтакте, на основе данных получаемых в автоматическом режиме. Состав дополнительного функционала, сообщающий значимую дополнительную информацию о человеке,  будет определен в процессе разработки, т.к. на начальном этапе не представляется возможным определить его из-за большого размера проекта социальной сети Вконтакте.

Оценка уровня конфиденциальности закрытых персональных данных пользователя на основе активности в социальной сети

\subsection{Цели и задачи дипломного проекта}
Задачи:
	\begin{itemize}
\item анализ лигитимности функционала приложения;
\item анализ существующих web-сервисов, которые предоставляют дополнительную информацию о пользователе с помощью анализа косвенных признаков;
\item анализ существующих научных подходов для реализация данной задачи;
\item составление описания для каждого решения;
\item анализ законности существования приложений данного типа;
\item анализ существующих научных подходов для реализация данной задачи;
\item реализация обязательного функционала. Уточнение и реализация дополнительного функционала;
\item тестирование и доработка приложения.
	\end{itemize}
	
\section{Анализ существующих решений}
Вследствие огромной популярности социальных сетей и того что они играют большую роль в жизни современных людей, уже давно стали появляться проекты, дополняющие их функционал. Продукты которые взаимодействуют с социальными сетями можно разделить на следующие категории:
\begin{itemize}
\item продукты для SMM - автоматизируют работы по привлечению внимания к брендам через социальные сети.
\item социальные агрегаторы - упращают упрвлению несколькими аккаунтами в социальных сетях. Как правило позволяют настроить крос-постинг, иногда присутствует функция собирание ленты, сообщений, комментариев с различных аккаунтов
\item сервисы анализа сообществ и трнедов в социальных сетях - позволяют вести SMM на более высоком уровне, проверять эффективность собственных pr-компаний и отслеживать конкурентов.
\item нативные приложения для социальной сети - дополняют функционал социальной сети, значительную долю приложений этого класса занимают игры.
\end{itemize} 
В анализ существующих решений вошли продукты, которые показали интересные технические решений, новаторство в отрасли или были интересны по другим причинам. Основной целью анализа  является сбор сведений о существующих решениях в отрасли и об общих тенденциях в развитии.  Анализ не ставил своей целью составить список лучших или всех приложений определенного рода. Стоит понимать что на данный момент количество сервисов и прилоежний у которые используют интеграция с социальными сетями уже сотни тысяч и описать их всех не имеет смысла\footnote{имеется ввиду все приложения, которые имеют функцию интеграции с какой-либо социальной сетью}
\subsection{smm-продукты}
Такие проекты автоматизируют задачи с использованием инструментария, предоставляемого непосредственно социальными сетями, например, публикация постов в определенное время, статистика популярности сообщений. Так же продукты этого класса могут автоматизировать любые другие действие упрощающие социальный медиа маркетинг (smm)\footnote{Social media marketing (\url{http://ru.wikipedia.org/wiki/Social_media_marketing})}
\subsection{социальные агригаторы}
Так же существуют проекты, программные продукты или сервисы, которые собирают информацию из разных социальных сетей, блогов и других ресурсов в один источник.\footnote{20 Ways To Aggregate Your Social Networking Profiles (\url{http://mashable.com/2007/07/17/social-network-aggregators/})} 
Стоит отметить что не все сервисы четко вписываются в тот или иной класс приложений, потому как многие из них достаточно самобытны и быстро изменяются и даже зачастую перестают существовать. Так за время подготовки дипломной работы перестали функционировать ряд сервисов:
\begin{itemize}
\item \subsubsection{twinfluence} был простым инструментом для измерения совокупного влияния твитов и их фолловеров, а также в качестве бонуса предоставляет статистику некоторых социальных сетей. В данный момент недоступен, по доменну на котором находился проект стоит переадрисацию на компанию в которой работают бывшие владельцы Twinfluence;
\item \subsubsection{TweetEffect} – отражал изменение количества фоловеров после каждого сообщения. Сервис перестал работать после изменения в twitter API;
\item \subsubsection {Tweetoclock.com} - помогал отследить время использования пользователями своего твиттер-аккаунта. В данный момент недоступен.
\end{itemize}

К самым интересным социальным агрегаторам можно отнести:
\begin{itemize}
\item \subsubsection{Hootsuite} - Один из самых надежных и доступных инструментов,  HootSuite постоянно совершенствует свой интерфейс и возможности. Онлайн-доступ позволяет войти в свой аккаунт с любого места, чтобы контролировать свои аккаунты. В настоящее время, есть поддержка Twitter, Facebook Pages, Facebook, LinkedIn, Ping.fm, Wordpress.com, MySpace и Foursquare. HootSuite обладает функционалом, которые позволяют настроить, отправку поста во множество источников в несколько кликов.\footnote{7 Social Media Aggregation Tools To Simplify Your Streams  \url{http://socialmediatoday.com/SMC/192312}} Ключевыми характеристиками являются:
	\begin{itemize}
	\item  Планирование. Выбор между обновлением постов он-лайн или по зарание загатовленному расписанию.
	\item Гибкая работа с url. Добавление ссылок-счетчиков для отслеживания кликов и получение детальной информации об аудитории.
	\item  RSS канал. Возможность добавить отправку постов в блоги и социальные медиа по RSS каналу.
	\item Закладки и аплет для браузера. Возможно использовать фирменный аплет для браузера, что бы быстро поделиться информацией
	\end{itemize}

%\item \subsubsection{Tweetdeck} - популярное бесплатное кроссплатформенное приложение, предназначенное для работы с социальными сетями. На данный момент является официальным клиентом Twitter. 
\end{itemize}
\
%more examples

Для данного исследование представляется наиболее важным выделить существующие методы получения информации и поиска в социальных сетях, в то время как остальные особенности сервисов отходят на второй план. Был проведен анализ существующих решений, выделен ряд приложений которые с помощью косвенных данных и методов автоматического анализа позволяют «вычислить» дополнительную информацию о человеке, которую он не указывал в явном виде, найти на web-ресурсах информацию не доступную обычным поисковым системам, получить релевантную информацию которая обычно слишком низко ранжируется. %! может сюда переставить кусок, о том что программы смешанного типа?

\subsection{web-приложения для поиска людей}
В сети Интернет представлен ряд приложений для поиска аккаунтов людей сразу во множестве социальных сетей. Стоит отметить, что в данный момент количество социальных сетей уже исчисляется десятками и это только те, которые имеют значительное (более нескольких миллионов) и живое сообщество.\footnote{Top 15 Most Popular Social Networking Sites (\url{http://www.ebizmba.com/articles/social-networking-websites})}
 Существует большое количество CMS,%!определение
конструкторов сайтов позволяющие достаточно быстро создать свою собственную социальную сеть или отдельный блог с интеграцией с другими блогами построенными на той же технологии.\footnote{8 Great Social Networking CMS (\url{http://www.cmscritic.com/8-great-social-networking-cms})} 
Все сети имеют свои особенности,  поэтому агрегация этого многообразия - задача не простая, и ее можно решить несколькими способами. К основным проблемам, которые необходимо решить таким приложениям являются:
\begin{itemize}
\item написание адаптеров для каждого источника информации\footnote{конечно существует Open API, но многие социальные сети имеют соци особенности, поэтому все таки необходим индивидуальный подход}
\item решение вопросов разряженности данных (социальные сети обладают различным функционалом и данными о своих пользователях)
\item скорость работы - агригатор собирает информацию с других сервисов и значит в вподает в зависимость от скорости работы 3-их лиц, что не всегда может быть надежно
\end{itemize}
основными представителями являются:
\begin{itemize}

\item \subsubsection{http://people.yandex.ru} %add picture
\url{people.yandex.ru} – это специализированная поисковая вертикаль, с помощью которой возможно быстро находить размещенные в открытом доступе профили людей в социальных сетях. Для поиска не требуется регистрация в социальных сетях. Характерной чертой является то, что сервис очень бережно относится к персональным данных пользователей:
\begin{itemize}
\item Не собирает и не хранит у себя никаких дополнительных данных о пользователе, лишь ищет и индексирует уже существующую информацию.
\item Индексирует только те профили, индексация которых не запрещена самим пользователем.
\item Индексирует только публично доступные данные, которые видны любому незалогиненному в социальной сети пользователю.
\item Склеивает только те профили, которые явно и публично ссылаются друг на друга (или в двух профилях проставлены взаимные ссылки друг на друга, или в одном из них есть провалидированная, т.е. требующая авторизации, ссылка на другой).
\end{itemize}
\item \subsubsection{http://qwant.com}
qwant.com — поисковая система с особым методами ранжированию и поиском по англоязычным социальным сетям (в этом она напоминает people.yandex.ru);
\item \subsubsection{http://spokeo.com}
spokeo.com — сайт для поиска людей, аггрегирующий информацию из множества других он-лайн и офф-лайн источников, таких как: телефонные справочники, социальные сети, фотоальбомы, маркетинговые исследования, списки рассылки, государственные переписи, безнесс-сайты, всего — более чем из 60 источников. Основные базы для поиска на английском языке и, как следствие, позволяет довольно точно отследить людей, пользующихся иностранными сайтами в повседневной жизни. Сервис является прекрасным примером того, насколько эффективным может быть автоматизированное использование различных источников данных.
\end{itemize}

\subsection{Сервисы анализа сообществ и трендов в социальных сетях}
В интернете содержится огромное количество книг, инструкций и примеров психологических анализов страницы из социальной сети, но сервисы для автоматизации этого процесса практически отсутствуют. Это можно объяснить тем, что на такого рода сервисы сложно манетизировать. Естественно, что у самихвладьцев есть подобные и даже куда мощные средства. Так например система матрикснет от Яндекс умеет классифицировать следующим образом пользователей.\\ %Ссылка на матрикс

Данный класс приложений похож на мое приложение тем, что с помощью автоматических алгоритмов  он анализирует состояние и изменения в сообществах и социумах, в то время как я анализирую отдельного человека. Некоторые из этих приложений уникальны и весьма интересны, и на основании этого включены в анализ. Интересно что много сервисов для анализа twitter`а являются некоммерческими и вследствии этого быстро теряли поддержку, так например в 2011 году эти сервисы еще существовали или были популяярны и хорошо работали:
\begin{itemize}
\item \subsubsection{http://topsy.com}
topsy.com - realtime поисковая система, специализирующаяся на поиске и аналитике по социальным медиа, таким как блоги, twitter, google+ и другие социальные сети. Компания является сертифицированным партнером twitter и поддерживает индекс всех сообщений начиная с момента создания twitter в 2006 году. Запуску предшествовали три года разработки. C 2012 года партнер Яндекс (используется в формирование новостной ленты), в 2013 куплена Apple за  \$200 мл. Ключевые характеристики:
\begin{itemize}
\item Анализ миллиардов разговоров в реальном времени.
\item Мгновенное получение новостей и информации об изменении в цитируемости
\item Поиск наиболее влиятельных пользователей Twitter по любой тематике
\item Просмотр продвижения любого хештега в Twitter. Возможность отследить искуственное раскручивание
\item Интерактивный анализ по ключевым словам и авторам, каталогизация по темам, влиянию, эмоциональной окраске, языку или географии. Пользователь может узнать, наиболее релевантные твиты, ссылки, фотографии и видео для любой терма из индекса Topsy в сотни миллиардов твитов. Пользователи могут групповые термы в сохраненных тем и настройки индивидуальных оповещений и ежедневных дайджестов деятельности.
\end{itemize}
Подводя итог, можно сказать что topsy - является одним из лидеров на рынке извлечения данных из социальных сетей, но в силу того что рынок чрезвычайно разнообразен и имеет множество особенностей в разных странах мира, то topsy не является едиственным представителем этого класса сервисов
\item \subsubsection{http://www.kribrum.ru/} - система мониторинга и анализа социальных медиа для управления репутацией в Интернете, позволяет отслеживать и анализировать упоминания бренда, продуктов, услуг и ключевых персон компании. Система в автоматическом режиме находит отзывы, обрабатывает их, определяет эмоциональную окраску высказываний и выгружает информацию в виде наглядных графиков и интерактивных отчетов. Интересно, что это одна из немногих отечественных разработок на этом рынке. Продукт принадлежит компании "Ашманов и партнеры"\footnote{Крибрум | Ашманов и партнеры \url{http://www.ashmanov.com/services/kribrum} }
	\begin{itemize}
	\item Широкий охват поиска. Порядка 700 000 отслеживаемых площадок, постоянно добавляются новые источники, в т.ч. по запросу пользователя
	\item Фильтрация спама, точность выборки. Система учитывает только те отзывы, которые относятся к объекту мониторинга, отсеивает спам и сообщения, в которых бренд упомянут вскользь.
	\item Автоматическое определение тональности и тематики сообщений
	\item собственная лингвистическая технология, которая позволяет системе «понимать» правила построения предложений, анализировать связи между словами и автоматически определять тональность высказывания (хорошо, плохо, нейтрально) относительно объекта мониторинга с точностью более 80%
	\item Оперативность обновления данных. Данные попадают в систему в период от 15 минут до 2-4 часов после публикации.
	\item Система позволяет определить общий охват, а также «вес» каждого упоминания и его автора, что особенно важно для формирования эффективной информационной политики, выбора подходящих площадок взаимодействия с аудиторией и выявления лидеров мнений.
Возможность реагирования
	\item Разнообразие отчетов, экспорт данных
	\item Автоматическая генерация отчетов по шаблону и рассылка по электронной почте по заданной схеме.
	\item Возможность заказать аналитический отчет у экспертов в области мониторинга социальных медиа.
	\item Ролевой доступ, система назначения заданий, журналирование действий операторов в системе\footnote{Что такое Крибрум \url{http://www.kribrum.ru/about/}}
	\end{itemize}


\item \subsubsection{TweetStats}
TweetStats - создает инфографику на основе постов человека в twitter по следующим направлениям:
	\begin{itemize}
	\item количество твиттов в час
	\item количество твиттов в месяц
	\item количество твиттов в зависимости от времени (день, ночь, день недели)
Есть функция сохранения результатов анализа.\footnote{TweetStats - Graph your Twitter Stats \url{http://www.tweetstats.com/}}. Проект особенно не развивается, масштаб проекта не большой, сервис просто хорошо справляется с заявленной функциональностью.
Tweets per month
Tweet timeline
Reply statistics
	\end{itemize}
показывает количество сообщений по месяцам, частоту сообщений в зависимости от времени дня и дня недели. Проект некоммерческий, не развивается, некоторые функции работают не стабильно;

\item \subsubsection{Twitteranalyzer}
Twitteranalyzer - статистика по  направлениям: Пользователи, Друзья, Упоминания, Группы и более мелким подуровням, что позволяет получить довольно много информации для анализа; Так же перестал работать. %!more text!
\item \subsubsection{sleepingtime.org} - Простой сервис с одно единственной функцией - определение времени сна по твиттам. Принцеип работы достаточно прост: сервис анализирует последние 1000 твитов и по ним строит приближенное расписание сна человека. Сервис обладает красивым интерфейсом и набором людей и областей из которых можно проанализировать людей, например шоу-бизнес, it-специалисты, политики, спортсмены.
\item \subsubsection{klout.com} - веб-сайт и мобильное приложение, которое использует аналитику по социальным медиа для выставления ранга от 0 до 100 под названием "Klout Score" по направлениям:
	\begin{itemize}
	\item True Reach - на какое количество пользователей вы оказываете влияние;
	\item Amplification - охват зоны влияния. Когда вы публикуете что-то, как много людей отвечает на ваш пост или перепечатывает его. Чем больше люди реагируют на ваши посты и сообщения, тем выше зона влияния;
	\item Network Score - как ваша аудитория реагирует на ваше влияние. Как часто пользователи (друзья, подписчики или их друзья) делятся вашим контентом со своими читателями и как далеко он расходится по сети? Чем больше вас упоминают, тем выше этот показатель.\footnote{Работаем с сервисом Klout - а как вы влияете на вашу аудиторию в социальных сетях? \url {https://www.facebook.com/notes/mike-ponomarenko/работаем-с-сервисом-klout-а-как-вы-влияете-на-вашу-аудиторию-в-социальных-сетях/233439080026653}}
	\end{itemize}
По заявлению разработчиков ранг является корреляцией между вкладом человека в контент социальных сетей и тем насколько контент, создаваемый пользователем востребован другими пользователями социальных сетей. Аналитика производится на основе данных сайтов Twitter, Facebook, Google+, LinkedIn, Foursquare, YouTube, Instagram, Tumblr, Blogger, WordPress, Last.fm и Flickr.\footnote{How can you measure influence? (\url{http://www.simplyzesty.com/Blog/Article/July-2010/How-can-you-measure-influence})}
\footnote{http://klout.com/corp/about \url{http://klout.com/corp/about}}
 Klout оценивает степень влияния, используя показатели такие показатели как:
	\begin{itemize}
	\item сколько авторов отслеживает пользователь;
	\item сколько авторов отслеживают пользователя;
	\item количество ретвитов
	\item упоминания в списках авторов;
	\item за сколькими спам/мертвыми авторами следит пользователь;
	\item какова степень влияния тех, кого ретвитит пользователь;
	\item количество приватных сообщений
	\end{itemize}
Полученная информация объединяется с информацией из Facebook, комментариями, отметками о понравившейся публикации, количеством друзей. Все эти данные отображаются в «Klout Score», который показывает степень влияния пользователя в социальных сетях.
У сервиса подвергается постоянной критики \footnote{Why Klout scores are possibly evil (\url {http://money.cnn.com/2011/11/15/technology/klout_scores/index.htm})}
\footnote{Don't Fall for this Sneaky Klout Trick Designed to Suck You In (\url{http://www.forbes.com/sites/anthonykosner/2012/05/08/klout-uses-this-trick-to-make-you-feel-bad-about-yourself-dont-let-it-ruin-your-life/})}
\footnote{Klout overhauls its business model, but does it answer its critics? \url{http://www.businessesgrow.com/2012/08/14/klout-overhauls-its-business-model-but-does-it-answer-its-critics/}}
 из-за того, во что он фактически получает власть человеческие судьбы, так в 2012 году в США одного за место специалиста с низким рейтингом klout взяли неопытного парня с высоким рейтингом.\footnote{см. подробней wired: What Your Klout Score Really Means (\url{http://www.wired.com/2012/04/ff_klout/})}. Джон Скалзи (John Scalzi) из CNN описал принцип, лежащий в основе Klout как «социально зло" в результате использования klout он вызывает тревожное состояние у своих пользователей.\footnote{Klout Now Measures Your Influence on Facebook \url {http://mashable.com/2010/10/14/facebook-klout}}
 Британский писатель Чарльз Стросс охарактеризовал klout как "герпес для интернета". Анализ условий использования и лицензионного соглашения klout показывает, что бизнес-модель компании является незаконной в Великобритании, где она противоречит закону  Data Protection Act 1998 года; Стросс советует читателям удалить их аккаунты Klout и отказаться от услуг этой компании.\footnote{Charlie Stross - Evil social networks (\url {http://www.antipope.org/charlie/blog-static/2011/11/evil-social-networks.html})}
%TwitterCounter - позволяет отслеживать статистику популярности вашего аккаунта, настроить уведомления себе на почту;\\
%Twittergrader - детальная статистика аккаунта с показами всех заходов и так далее;\\
%Klout - статистика аккаунта, динамика роста, его твит-сила;
%Tweetmetrics - разнообразная статистика аккаунта;
%Trendrr - мощный сервис для сбора статистики в интернете, включая Твиттер;
%Trendistic - дает представление о популярности того или иного запроса, представляет его количество повторений в Twitter на графике;
%TweetBeep - позволяет отслеживать упоминание бренда, имени, сайта, вообще любого слова в Твиттере, присылая уведомления;
%Analytics.ad.ly - систематизирует информацию о фолловерах;
%TwitterStreamGraphs - интерактивный инструмент, позволяющий создать график на основе слов упоминающихся в сообщениях пользователей Твиттере;
%
%Radian - мощное средство для анализа социальных сервисов, включающее в себя и инструменты для анализа Твиттера;
%http://www.tweetping.net/!!!
\end{itemize}
\subsection{Приложения для платформы vk.com}
Отдельно стоит упоминать приложения, написанные на платформе ВКонтакте — все они реализуются по средствам flash, javascript или как iframe со стороннего сайта. Особенностью приложений под эту платорму является их относительная простота, как правило выполняются на стороне клиента или имеют не сложную серверную по сравнению с приложениями для анализа сообществ и трендов. В ходе анализа существующих решений были выявлены следующие приложения:
\begin{itemize}
\item Анализатор (\url {https://vk.com/ianaliz}) — способен проанализировать количество друзей, сколько из них женского пола, сколько мужского, сколько не сообщили такую информацию, примерная дата регистрации в vk.com
\item Радар (\url{https://vk.com/vkradar}) — сообщают статистику по сообщениям на стене, в группе и т. д. по количеству сообщений, по полярности сообщений. Относительно не сложное приложение, с моделью монетизации за дополнительный функционал.
\item Модерация пабликов и страниц, анализ популярности (\url{http://vk.com/public_tools}) - инструмент для автоматизации повседевных задач модератора. Отличается низкой надёжностью (стало временно не доступно во время написания работы)
\item Анализ Аватара (\url{https://vk.com/avascan}) — выдает результаты близкие к случайным но на основе аватара пользователя по таким характеристикам как: сексуальность, красота и прочее. Не так давно было заблокировано администрацией платформы.
\item Лайк-машина (\url{http://vk.com/like.machine}) приложение, позволяющие за дополнительную плату повысить свою популярность и отследить посетителей.
Больше приложений для платформы vk решющих подобные задачи выявлено не было
\end{itemize}
По итогам анализа были сделаны следующие выводы:
	\begin{itemize}
	\item представляется интересным показывать результаты работы приложения посредствам публикации отчета на стене пользователя
	\item возможна монетизация приложения путем предоставления пользователю бесплатно, некоторого количество виртуальных денег, которых хватит, на то чтобы попробовать лишь часть функционала, в то время, как весь функционал будет стоить дополнительных денег.
	\item В приложение можно реализовать дополнительный(не основной) функционал, но он должен пониматься пользователем как дополнительный и предоставляться бесплатно
	\end{itemize}
\subsection{Выводы}
По итогам анализа этих проектов были сделаны следующие выводы:
\begin{itemize}
\item большие объемы данных позволяют построить более детальную аналитику, чем локальный анализ
\item так как в разных социальных сетях сидят одни и и те же людей, то при отслеживании каких либо общественных изменений, как правило, достаточно глубоко анализа одной из платформ, поэтому большинство сервисов заточены на Twitter, как наиболее удобную и открытую социальную сеть из всех.\marginpar{Несогласованное предложение?}
\item достаточно интересным оказался функицонал сайта  sleepingtime.org, который анализирует время публикации постов. В дальнейшем возможно развить отсюда следующие направления:
	\begin{itemize}
	\item вычислить время, в которое пользователь активно пишет посты в социальной сети
	\item вычислить время сна
	\item примерно вычислить сколько часов в день пользователь проводит в социальной сети
	\end{itemize}
\textcolor{red}{я точно не реализую эту функциональность, стоит ли тогда писать об этом? может как то изменить предложение?)}
\item показательным является пример klout, который зарабатывает деньги меняя наше общество к худшему
\end{itemize}
Среди всех приложений я пытался выявить приложения которые анализируют метаданные в целях получения дополнительных сведений о пользователи. Таких приложений оказалось немного: sleepingtime, Радар, Анализатор для vkontakte. Приложения Радар и Анализатор не реализуют функционала который заложен в дипломный проект. Фактически такого рода сервис до этого не кто не делал, во всяком случае широкой общественной в русскоязычном сегменте интернета это не известно. По всей видимости,  малое количество приложений можно объяснить тем, что вся мощь заключенная в метаданных раскрывается при больших объемах информации, создание приложений анализирующих большие объемы стоит больших денег поэтому перед созданием приложения должна существовать ясная модель монетизация приложения. В гражданской сфере в основном востребованным мониторинг брендов, особником стоит относительно новый сервис klout. Совсем другая ситуация складывается в военном секторе, особенно в области разведки, где созданы огромные системе, такие как Xkeyscore и другие, в которых основной источником анализа являются метаданные.  %!more!

%!more from english sites!!!
Подводя итог можно сказать, что не было выявлено приложений реализующих 

\section{Обоснование выбора технических инструментов для реализации приложения}
Приложение планируется создать по клиент-серверной архитектуре. Сервер обработки данных на Python, база данных на MySQL, веб-форма на фреймворке Django. Так как планируется использовать отдельный сервис для обработки данных, то размещение проекта на виртуальном (shared) хостинге является недостаточным. Проект будет размещен на виртуальном выделенном сервере (VPS), ОС для сервера — Ubuntu 12.04 \\
Все инструменты, которые я применяю, является продуктами с открытым исходным кодом. Они активно развиваются, поддерживаются и имеют живое сообщество пользователей в т.ч. русскоязычное.\\ 
Python выбран в связи с большой функциональной выразительностью1 и гибкостью языка2. Проект не рассчитан на очень высокие нагрузки, поэтому с одной стороны Python-а вполне достаточно, а с другой упрощается процесс написания и сопровождения приложения. \\
MySQL выбрана как одна из самых быстрых СУБД при средних и маленьких объемах БД. Так же у проекта хорошая документация.\\
Django выбран за высокую скорость написания приложения и архитектурные преимущества, по сравнению с такими фреймворками как Symfony(PHP) и Dancer (Perl), а также в связи с тем, что написание приложения обработки данных и приложения веб-клиента на одном языке упрощает сопровождение (Django написан на Python).\\
Ubuntu server 12.04 выбрана среди прочих аналогов, таких как Fedora, Debian, OpenSuse по следующим причинам:\\
	\begin{itemize}
	\item дружелюбное сообщество
	\item безопасность
	\item легкую расширяемость с помощью ppa1 и центра приложений
	\item активная поддержка (Ubuntu server 12.04 будет поддерживаться до апреля 2017 года)1
	
	В процессе написания приложения необходимо будет решить следующие проблемы:
	\item распознавание информации с сайта
	\item обход страниц друзей пользователя и распознавании информации на их странице
	\item минимизация нагрузки с одного клиента
	\item авторизация пользователя средствами социальной сети (по протоколу OAuth 2.0)
	
	\end{itemize}

безопасность1
легкую расширяемость с помощью ppa2 и центра приложений
активная поддержка (Ubuntu server 12.04 будет поддерживаться до апреля 2017 года)3

