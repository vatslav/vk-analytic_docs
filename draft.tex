
%Другим интересным фактом, который был замечен, исследован и добавлен в програмную часть \marginpar{не забыть убрать, если не успею} анализатора анкеты является, то что пользователь чаще всего упоминает в своих сообщениях город в котором живет, то же касается и возраста - как правило пользователи пишут людям своего возраста, а так же теми с кем они вместе работают или учатся.

%Автор предположил, что пользователи чаще всего добавляют в друзья людей своего возраста, живущих с ними в одном городе. Так же, если проранжировать по частоте встречаемости учебные заведения в которых учатся друзья пользователя, то наиболее часто встречается то учебное заведение, в котором учится пользователь. Что бы подтвердить данное наблюдение было произведено исследование тысячи случайно взятых людей из социальной сети Вконтакте. Подробнее это исследование будет рассмотрено в следующих главах.\\
 
%Таким образом, вторым косвенным признаком который будет анализироваться дипломным продуктом будут являться комментарии и записи пользователей.
%

smm - нужны?


ст 148 надо наверное вставить фразу типа - характерные особенности для данного пункта
ст 178 каким набором людей?

ст 201 взяли куда?

ст 217 серверную что?

ст 227 и ст 234 это  один анализ или 2 разных - уточнить